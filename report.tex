\documentclass[11pt]{article}
\usepackage[a4paper,pdftex]{geometry}
\setlength{\oddsidemargin}{5mm}
\setlength{\evensidemargin}{5mm}
\usepackage[english]{babel}
\usepackage{amsmath,amsfonts,amsthm,amssymb}
\usepackage{graphicx}
\usepackage{fancyhdr}
\pagestyle{fancy}
\usepackage{url}
\usepackage{lastpage}
\urlstyle{same}

% Page numbering
\lhead{SmartClass: Optimizing Education}
\rhead{page \thepage/\pageref{LastPage}}
\cfoot{}
\rfoot{\thepage}

% TITLE FORMAT
\newcommand{\HRule}[1]{\rule{\linewidth}{#1}}

\makeatletter
\def\printtitle{
    {\centering \@title\par}}
\makeatother                  

\makeatletter
\def\printauthor{
    {\centering \large \@author}}
\makeatother

% TITLE
\title{
\HRule{0.5pt} \\
\LARGE \textbf{\textsc{SmartClass}}\\[0.5cm]
\normalsize \textsc{Optimizing Education}
\HRule{2pt}\\ [0.5cm]
\normalsize
\today
}

\author{
\vspace{0.5cm}
Supervised by dr. F. Nack and M. S. Latour\\[0.5cm]
\begin{tabular}{c c c c}
M. A. Cabot & S. Laan & C. R. Verschoor & A. J. Wiggers\\
6047262 & 6036031 & 10017321 & 6036163
\end{tabular}\\[0.5cm]
Artificial Intelligence\\
Faculty of Science\\
Universiteit van Amsterdam\\
}

% BEGIN DOCUMENT
\begin{document}

% TITLE PAGE
\thispagestyle{empty}
\printtitle                  
\printauthor
\vfill
\begin{abstract}
\noindent This paper describes a design of a knowledge media application for universities based on the user experience of lectures in university. The application, called SmartClass, gathers data from the user, internet and previous presentations in order to create new presentation slides based on the feedback of users. The application focuses on the individual experience of the users: the teacher and the student. All processed data is represented in RDF format, to provide a reasoning logic that can run queries in our ontology and browse through the available data. The user can ask questions about specific elements of the presentation. The application searches the information about this query and presents this to the user on newly generated slides. The application contains a self learning system that improves the presentation based on feedback, namely the approval or denial of slides and the amount of questions asked.
\end{abstract}
\newpage

% TABLEOFCONTENTS
\setcounter{page}{1}
\tableofcontents
\newpage

% CONTENT
\section{Introduction}
This paper shows the framework of a knowledge based multimedia system designed to generate new slides for a presentation in two scenarios. The first scenario is when the system generates new slides for the user that gives the lecture, offline, improving the presentation that was used the year before. The second scenario is when the system generates a personal slide, online, for a student who has a question about the given lecture. In this case offline means that generation of presentation slides happens before the presentation and online means that the application generates real time the presentation slides. This application was designed with the perspective of the users, the teacher and the students, in mind.\\\\
SmartClass was designed with the following in mind: `What would the student expect from a system that enhances lectures?'. The system should be able to make a lecture a more complete and more interesting learning experience than a regular lecture. Not only should it encourage students to actively participate and ask questions, it should also enable them to give feedback on the slides and presentation indirectly. \\\\
During a lecture, students may have a question about the content of the presentation. Several problems can arise when it comes to asking questions during a lecture:
\begin{itemize}
\item Students may fear interrupting the teacher.
\item Students may think that their question is not a very good one.
\item Students may think that their question involves something that has already been answered earlier in the lecture. 
\item A teacher may be interrupted by questions that involve earlier parts in the slides (if the student was not paying attention).
\end{itemize}

SmartClass will make the experience of asking questions anonymous and simple, resulting in a better lecture experience.\\\\
Many teachers use the slides of previous years as a guideline for the next year, or sometimes even copy them entirely. In the first case, it will cost a significant amount of time to adjust the slides manually. In the second case, the slides may be outdated, new techniques or discoveries might have been made that require the slides to be adjusted. In both cases, SmartClass comes in handy.  Generation of new slides happens based on slides of previous years but the system looks at new sources simultaneously. These slides can still be fully edited giving the teacher full control over the content of his/her presentation. \\\\

In the next sections the following aspects regarding the application are described: the user groups, the design and concepts, the application from student perspective, the application from teacher perspective, the interface of the applications and lastly the application is discussed. 

\section{User groups}
The system is designed for two specific groups of users, both of which will be described in this section.
\subsection{Students}
The first user group is the group of students. Students will only use the system during lectures. 
Since no two students are the same, the following three `stereotypical' students were created to take into account as much students as possible :
\begin{itemize}
\item The Normal Students pay attention during lectures. When they have a question it will be an in-depth question. They want to know more than is necessary.
\item The Lazy Students are lazy. These students do not take notes, do not pay attention, or at least not the whole time. In general, they will not ask any questions. The only exception is when a subject is discussed that appeals to them personally. They might also ask sarcastic or irrelevant questions.
\item The Not That Smart Students are a bit slower of understanding. They always are a few steps behind. Whenever they ask a question, it might have already been answered, they just did not notice the answer. Other questions from these students include questions about concepts that are considered known, but they forgot.
\end{itemize}

These three types of students were taken into consideration in the design process.

\subsection{Teachers}
The second group of users consists of teachers. This group should have more control over the system than a student. It is important that the system gives teachers this control without distracting them from their main task, namely, giving the lecture. No distinction was made between different types of teacher, but the system should be able to `learn' the preferences of the teacher through examples and feedback.  


\section{System overview}
The application will allow users to access new generated content in two ways: Firstly, by generating new presentations based on previous presentations. Secondly, by generating a new presentation slide based on the question asked by the user. This section describes the concepts that are used to design the SmartClass application.

The system that was designed consists of several components. In this section each component will be described in detail, but first an overview of the whole system is given: the global interaction and system flow are described. First the global system flow is explained.

Figure \ref{flowchart} shows the global system flow. Firstly, the teacher has to input some information. This information is used in the presentation generation. The slides the system produces have to be evaluated by the teacher. If the teacher does not approve, he has to give feedback and the system tries again.
When the slides are approved, the presentation can commence. In practice there will be much more time between the approval of the slides and the start of the presentation; the flow chart gives the impression this happens immediately after the approval, but this does not have to be the case.
When the presentation starts, the students have the opportunity to interact with the system. They can ask questions to the system. In the event this happens, the system tries to answer the question and generates slides to answer it.

\begin{figure}[!h]
\centering
\includegraphics[width=0.7\textwidth]{KBMSFlowchartnew.pdf}
\caption{Flowchart of the system.}
\label{flowchart}
\end{figure}

In figure \ref{systemOverview} the different components of the system are displayed. The arrows indicate communication links between components. The global interaction between these different system parts will be explained shortly. 

\begin{figure}[!h]
\centering
\includegraphics[width=0.7\textwidth]{systemOverview.pdf}
\caption{The separate parts of the system and how they are connected.}
\label{systemOverview}
\end{figure}

The main concept the system uses is the slide generation component. The same slide generator is used for the online and offline generation. The input for the slide generation comes from different sources. The main source is the external knowledge base. The other inputs come from the students (via the Quick Question Dialog) and the teacher (via pre-generation). Finally the knowledge that is necessary for actually creating the slides comes from the system knowledge base.
When the slides are generated they are stored in the system knowledge base. The question with the answer slide will appear in the Quick Question Box, for both teacher and student.
The Quick Question Dialog uses the external knowledge base for definition questions, this will be explained in more detail later. The system knowledge base provides the tools to convert the posed question into a RDF query.


\subsection{Lecture representation}
The lecture representation holds the framework of the presentation. It defines how the course material is arranged. Firstly, the system must decide the main structure of the lecture. For example; first the introduction, then content and finally the conclusion. Secondly, the system must decide the flow of the lecture. A possible flow could be to first start with explaining the important terms and then use an example to clarify them even more.

Once the main structure and flow is determined the system must order the information that can be used for the lecture. In a good lecture each slide only contains terms and topics that are:
\begin{itemize}
\item known to the students in advance. This could be because it was explained in a previous lecture or because it is assumed to be common knowledge.
\item explained in the previous slides of this lecture.
\item explained in the current slide of this lecture.
\end{itemize}
The system must thus understand that to understand term A some other terms B and C must be understood first. It would then make no sense to place the slide explaining term A before the slides explaining term B and C. In order to know that B and C explain A the system contains hierarchical knowledge about the relations between the terms and topics of the lecture.

\subsection{Slide representation}
Each slide (or combination of slides) should contain the following components:
\begin{description}
\item[The title.] It addresses the main topic of the current slide.
\item[Bullet points.] Each bullet point is related/linked to the title. 
\item[Text.] Each piece of text is related/linked to a bullet point.
\item[Images.] Each image is related/linked to a bullet point.
\end{description}
The system knows what the relationships are between the components on a slide. In case a question is asked about a term, the system must know what type of components would be best as answering the question. Some questions would be better answered with a picture (e.g. `What is an apple?') while for other questions text would be more suitable (e.g. `When was king Louis XVI born?'). In order to do so, the system must have knowledge about where the term lies in the hierarchy of objects (e.g. is the question about the appearance of the term or about a date).

\subsection{Slide generation in general}
Each slide is generated according to a template. The components are all filled in in the correct place. The system can choose between multiple templates, based on the number (and type) of components that are to be used. 

There is a limit to the number of words on a single slide to prevent slides from becoming walls of text. The system will create multiple slides if a single one is not sufficient. If there is to much information that can be displayed on a single slide, then the system will prefer to split the information after a single component, e.g. at the end of bullet points. The system would not just simply generate a new slide when the old one is full. 

The system also prefers balanced slides over slides filled with a single component, e.g. two slides, each containing two images and two pieces of text, will be preferred over a slide containing only text and a slide containing only images. How this balance is reached depends on the semantic knowledge about each component. The system will try to distribute components in such a way that corresponding ones are put on the same slides, using the semantic knowledge it has of these components. 

\subsection{Offline generation}
Offline generation is the act of generating new slides for the presentation based on the previous slides. This action is performed before a lecture so that the teacher knows what the slides will look like. Depending on the representation of the previous presentation different action need to be performed during the offline generation. 

The first time a lecture is given using this system the previous slides do not have the required SmartClass-representation. In this case the system must place the content of the old slides in the representation used by the system. The system tries to match its lecture and slide templates with the slides of the previous slides. The templates that match best are then used for the offline generation. If the previous slides already have the required SmartClass-representation, then the previous templates are simply used for the offline generation.

SmartClass uses a set of heuristics to determine whether to change the structure of the lecture. Changing the structure would be based on the feedback (if such exists) of last year's lecture. If last year the teacher chose to use a slide generated to answer a student's question, then this time the system can choose to add the slide to the presentation. Another set of heuristics are used to determine whether to update the content of the of a slide. The systems needs to search the external database to see if a piece of text, image or other slide component is outdated. If so, the system will replace these components (taking into consideration the lecture and slide representation).

When SmartClass is finished with generating slides the teacher is able to reject, alter or add new slides. This way the teacher is always in full control of what the slides will depict. 

\subsection{Online generation}
A student is able to ask a question about every element of a slide by clicking on it. Online generation is the act of generating a new slide in real-time that answers the student's question. We distinguish between two types of questions: definition and open questions. 

A definition-question is a question about the definition of a term. The system create a slide that has as title the selected term and as content the definition of that term. If available, an image can be added to clarify the definition. This type of question will be especially helpful for the forgetful and less intelligent student. 

An open-question is any question other than a definition-question. Of all the information available about the term in question, only the information that is relevant to the question needs to be displayed on the new slide. A keyword-extractor decomposes the question into its relevant components: the subject, relation and object. One of these parts is the selected term (what the question is about) and the other parts are related to the question that needs to be answered. What essentially happens is that the question is translated into a RDF query. A RDF query can then be used to find the relevant information on the external database. After retrieving the relevant information a slide can be generated according to the lecture and slide representation explained above.

\subsection{Information Gathering}
The application gathers information from the following sources:
\begin{description}
\item[Student user of the application.] The amount of questions users ask concerning a presentation slide indicates the value of the slide (feedback to the application).
\item[Teacher user of the application.] Whether a teacher presents or deletes a newly generated presentation slide (feedback to the application).
\item[Encyclopedia or content related websites.] To automatically generate a new presentation slide containing the answer to the question.
\item[Previous generated slides.] To generate new presentation slides based on the performance of previous presentation slides.
\end{description}

\subsection{Information Searching}
A search for information is necessary if new slides are to be generated. When a question is asked, there are two important features for this question.
\begin{itemize}
\item The domain of the question. This is determined by the slide at which it was asked. 
\item Textual input. Keywords in the question (e.g. `where', `when', `how') and the use of known, relevant concepts (e.g. a question about a cockpit in a presentation about airplanes) are the most important.
\end{itemize}
When a new presentation is generated the application uses two important features of the slide to finds new content related to the slide.
\begin{itemize}
\item Domain of the slide. This is determined by the previous presentations and its feedback, plus the keywords of said previous presentations.
\item Meta-information, such as the form of a slide, the maximum number of images or text on a slide, etc.
\end{itemize}

\subsection{Self learning system}
The system will use machine learning methods to improve itself. There are multiple features that can tell the system how well it is functioning:
\begin{description}
\item[Student questions.] Several questions for one slide indicate that this slide is not very clear and possibly that the slide needs additional information about the subject(s) of the questions. 
\item[Approval by teachers.] Since the teacher can approve or reject newly formed slides, the system can learn which slides are favorable. Of course, this does lead to a biased idea of a good slide, since the teacher may have a wrong idea about slide creation. 
\item[Test results.] When several students make mistakes concerning one particular subject, it needs extra coverage. The system should be able to provide additional slides about this particular subject.
\end{description}

\subsection{Login system}
The application uses the Central Authentication Service (CAS) of the University of Amsterdam to allow users to access the application. This enables the system to save the users personal data (such as notes or question) without having to register on our application. CAS allows web applications to authenticate users without gaining their security credentials. Since every user of the application has an account at the university, this service fits in our application. Figure \ref{CAS} shows a picture of the CAS login screen.
\begin{figure}[!h]
\centering
\includegraphics[width=0.7\textwidth]{Cas.png}
\caption{Central Authentication Service login screen}
\label{CAS}
\end{figure}

\section{Student Perspective}
\subsection{Student interface}

\begin{figure}[!h]
\centering
\includegraphics[width=0.7\textwidth]{studentInterface.pdf}
\caption{The student interface. The students have the option to ask a question anonymously.}
\label{studentInterface}
\end{figure}

\subsubsection{Main Window}
The main window is where the most time is spent. It consists of the following components shown in figure \ref{studentInterface}.
\begin{description}
\item[1. Quick Question Box] In the Quick Question Box the best questions are shown together with the number of votes. The questions can be clicked. If SmartClass was able to generate a slide to answer this question, then this slide is shown. Each question is accompanied by two buttons that allow a user to vote in favor or in disfavor of the question. 
\item[2. The Presentation Panel] In the presentation panel slides are shown. The keywords on the slides can be clicked. On clicking the Question Dialog will popup. By default, the slide on the presentation panel automatically synchronizes with the slide being discussed by the teacher. An exception is when a student is making notes about a slide. In this case the panel will synchronize after the student is done making notes. It is also possible to toggle the synchronization off or on manually.  
\item[3. Navigation Thumbnails] The navigation thumbnails are little previews of the slides to come and the slides that already have been showed. They can be used to navigate through the presentation.
\item[4. Notes Textbox] In the Notes Textbox, the student can make notes. These notes will be linked to the slide that is currently shown (in the presentation panel, which is not essentially the same as the slide the teacher is showing). The textbox is scrollable. When moving from one note to another, the slides will move accordingly. 
\end{description}

\subsubsection{Question dialog}

\begin{figure}[!h]
\centering
\includegraphics[width=0.7\textwidth]{studentInterfaceDialog.pdf}
\includegraphics[width=0.7\textwidth]{studentInterfaceAnswer.pdf}
\caption{The student interface. The first image shows a question being asked about a component on the current slide. The second image shows a question being selected in the QQB. The students have the option to ask a question anonymously.}
\label{studentInterfaceDialogAnswer}
\end{figure}

The question dialog is a popup window in which the student can type a question or hit the definition-button (see figure \ref{studentInterfaceDialogAnswer}). There are various ways to influence the output of the system. Every user that is a student has three options:
\begin{description}
\item[Asking questions.] A visual change in output is that the question will appear in the question box. The system will also try to find additional information about the subject that might help answer the question.
\item[Voting in favor or in disfavor of questions.] A voting system is used to determine the importance of unanswered questions. Users can vote for a question if they also want additional information about the subject. The questions with the most votes appear at the top of the question box to indicate that these have the highest priority.
\item[Making notes.] As these notes are personal, they will not change the output for other users. When the user is taking notes of a certain presentation slide, the student can go back to the concerning presentation slide.
\end{description}

\subsubsection{Notes}
As stated before, beneath every slide students are able to make their notes. After a lecture the students will be able to extract a file containing the slides with their notes. 


\section{Teacher Perspective}
Many teachers use the slides of previous years as a guideline for the next year, or sometimes even copy them entirely. In the first case, it will cost a significant amount of time to adjust the slides manually. In the second case, the slides may be outdated, new techniques or discoveries might have been made that require the slides to be adjusted. In both cases, SmartClass comes in handy.  Generation of new slides happens based on slides of previous years but the system looks at new sources simultaneously. These slides can still be fully edited giving the teacher full control over the content of his/her presentation.

The presentation panel and speaker notes of the teacher interface are designed to be intuitive and similar to the way `normal' presentations are given. In addition, the teacher now has the QQB allowing him/her to be aware of pressing questions by students. Again, the teacher is in control of the content being shown. SmartClass leaves it up to the teacher to decide whether a newly generated slide is to be displayed. 

\subsection{Teacher Interface}
\subsubsection{Main Window}

\begin{figure}[!h]
\centering
\includegraphics[width=0.7\textwidth]{teacherInterface.pdf}
\caption{The teacher interface. The interface displays the most important and unanswered questions.}
\label{teacherInterface}
\end{figure}

Figure \ref{teacherInterface} shows the teacher's interface during a lecture.
\begin{description}
\item[1. Quick Question Box] By default, the question box looks the same as that of the students. The difference is that when a teacher clicks a question, it will lead him to a preview of the slide which he/she can then choose to incorporate in his/her presentation by clicking the `accept' button. 
\item[2. The Presentation Panel] Just as in the students interface, the presentation panel shows the current slides. Since the teacher is the one in charge of scrolling through the slides, there is no option for synchronization. 
\item[3. Preview box] In this box, the teacher will be able to see a pre-generated slide after clicking a question. The teacher is able to include one of these slides in his/her presentation. 
\item[4. Notes Textbox] In the Notes Textbox, the teacher can see personal notes that he/she made beforehand. It is usually easier to give a presentation if some keywords are written down beforehand, synchronizing them with the slides only makes it easier. The teacher does not have the ability to make new notes during the presentation, only beforehand.
\end{description}
\subsubsection{Editor window}

\begin{figure}[!h]
\centering
\includegraphics[width=0.7\textwidth]{teacherInterfaceEditor.pdf}
\caption{The teacher editor. As opposed to the student interface, the teacher interface does not display the questions anonymous.}
\label{teacherInterfaceEditor}
\end{figure}

Figure \ref{teacherInterfaceEditor} shows the interface of a teacher when generating a lecture offline. A teacher has the following options to influence the system: 
\begin{description}
\item[Deleting generated slides.] A teacher has to approve of each slide before they are used in the presentation. This is to avoid confusion and to leave out irrelevant and unnecessary information.
\item[Using generated slides.] When a teacher decides that he/she will use the generated slide, it is incorporated in the presentation. 
\end{description}

\section{Quick Question Box and its importance}
The Quick Question Box (QQB) is displayed on both the students' interface and the teacher's interface. As stated before the QQB contains a list of questions that have been asked by the students. Each question is linked to its generated slide (only if an answer could be generated). This gives students, especially lazy students, easy access to extra information without having to type the question themselves. As stated before, each question can be voted up or down and are sorted by their number of votes. This gives the teacher an overview of the questions that are puzzling most students. The teacher can then choose to cover a question with the whole class using the slide generated to answer this question. The teacher has also the option to deny a slide, when it is not covering the question at all in order to give feedback to the system. 

This question box is in fact (apart from real life contact of course) the only way for the teacher to interact with the students using the system. It can be done directly, by answering the questions himself or incorporating the slides into his or her own presentation, or indirectly by approving or disapproving of slides that were generated to answer questions. 



\section{Discussion}
\subsection{Improvements}
The system can be improved in a number of ways. Some of the shortcomings and possible adjustments of the current design will be discussed. 
\begin{itemize}
\item Currently, if a student asks a question using the system, a slide is generated (if the system can give an answer) and then shown to the student. The problem here is that there is no guarantee that this answers the question, nor that it is the correct answer. The system may have somehow found false information or parsed the question wrong if it is an open question, thus creating the wrong query. Since the system will give the answer slide to the student during the presentation, there is no way for the teacher to check each and every slide that is generated. This manual check can be made afterwards but it would cost extra time and more importantly: it would be too late, the student already has the false knowledge. 

A solution for this problem would be involving the other students in the slide creation. One approach to this is that when a question is asked, a slide will still be generated, but this slide can also be given votes by the students. Another approach is that students are able to answer questions themselves, but this may lead to chatting behaviour. One of the better systems that uses this approach (and has been using it successfully) is StackOverflow \footnote{www.stackoverflow.com}, an online point for questions related to programming. A user can ask a question and this question will receive votes if other users think it is a good question. Then, anyone that is logged in can answer this question, and these answers can also receive votes if they are good. A user that asks good questions or gives good answers receives `reputation' as reward. This reputation is linked to the corresponding account, showing how credible the user is. However, this site is moderated to reduce irrelevant questions and answers and it is possible to answer questions quite some time after they have been asked. The system will only be online during the lecture and there are no moderators available, making it hard for such a system to work correctly. 

\item If there are students in class that do not take the lecture, nor the lecturer, too seriously. They ask the system a question that is irrelevant to the lecture, for example: a lecture about instruments would lead to the question `Is mayonnaise an instrument'. Other students find this question so funny that they vote in favor of it. The teacher sees this question pop up and depending on the reaction, this behaviour will either stop or continue. Since the questions are asked anonymously, noone gets caught. A solution for this is making names visible. This was also incorporated in the current design. However, part of the problem remains. Imagine the situation where the teacher calls out the student for asking such an irrelevant question. Would the student feel like he is scolded or would he simply be getting the credit for asking something funny?

\item The students are able to save their slides alongside the notes they made for the presentation. It is still unknown if this will work better than regular notes, equally good, or worse. Perhaps, for some students, making notes is hard if they already have the slides in front of them and are too busy with reading them to make notes. It is also not sure if the students will not be distracted by the question box.  There is currently no solution for this problem yet, perhaps a configuration file for each user, containing their preferences and which components should be shown and which should not be shown, would be useful here. 
\end{itemize}


\subsection{Future work}
There are some additions that could contribute to the system. Some of the techniques that would be used to create such additions already exist, it is just that they were not taken into this design to prevent it from becoming too broad of a subject:
\begin{itemize}
\item 
Currently, the system only receives feedback via buttons on the screen. It would also be possible to extract more features by adding camera's and audio input devices. Not only would the system then be able to capture the lecture entirely, asked questions (in real life) can also be recorded. If these were parsed, the system could create a slide for these questions as well. 
\item
As stated above, user information for each user (including students) would also be useful. A screen giving options about what components to show would contribute to the idea that the system is not designed for a specific type of student.
\item 
The ability to answer questions asked by other students can be very useful. A teacher does not have time to answer all questions that are asked, which is where the voting system comes in handy. However, if a question is asked and does not get enough votes because the answer is simple, it would be a good contribution to the system if other students could give answers if the system can not answer the question or not answer it correctly (for whatever reason). Plus, there is also the possibility of students using this question-response structure to chat with eachother. 
\end{itemize}

%TODO fisnish
\section{Conclusion}
SmartClass is a system that enhances the lecture experience for both the students and the teacher. Students no longer have to be afraid of asking questions and can be helped individually by the system without interrupting the whole class. Before the lecture the teacher has to put little effort into generating or updating the slides of last year's lecture. During the lecture the teacher is advised on what part of the lecture is unclear to a large part of the students. SmartClass helps the teacher helps the teacher in such a way that still leaves him/her in control of what happens. 

Although the system is not foolproof, it is still a big improvement on traditional lectures. Not only will the time to create presentations decrease significantly, the risk of using outdated presentation material decreases as well. Students are able to save their notes together with the corresponding slides, maybe this will work better than making notes only (though this depends on the student, of course). 




\end{document}
​

